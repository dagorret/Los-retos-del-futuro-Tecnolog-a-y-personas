\documentclass[a4paper, 12pt]{article}

\usepackage[spanish]{babel}
\usepackage[utf8]{inputenc}
\usepackage{graphicx}
\usepackage[colorinlistoftodos]{todonotes}

\title{Los retos del futuro: Tecnología y personas}

\author{Carlos M. Dagorret}

\date{\today}

\begin{document}
\maketitle

\begin{abstract}
En un mundo de vertiginosos cambios como el que estamos viviendo, la tecnología está cada vez más presente en nuestras vidas y en todo tipo de organizaciones, sean ellas empresas de producción y comercialización de bienes de consumo masivo, instituciones educativas, instituciones dedicadas al cuidado de la salud, ejércitos, empresas de transporte, agrícolas, de desarrollo de software, de provisión de servicios de comunicaciones, etc.

La sociedad de la información y de las nuevas tecnologías se ha expandido a todos los campos de la ciencia, de allí que hoy se hable de biotecnología, de nanotecnología, de tecnología informática y de comunicaciones, de tecnología educativa, etc.

Ya vivimos en la era digital en la cual el tamaño y la ubicación física están cediendo paso a la virtualidad y en la cual los robots se están haciendo cargo de muchos de los trabajos que antes tenía el ser humano. Más aún, la inteligencia artificial además les posibilita solucionar problemas que se le presentan. Los robots no solo actúan como humanos sino que también piensan por nosotros.

El presente trabajo tiene un fin descriptivo de los elementos en conflicto en una organización y la tecnología de la información.
\end{abstract}

\section{Introducción}
\label{sec:Introduccion}

La palabra tecnología viene del griego tekhne (arte, técnica u oficio) y logos (estudio, tratado). De allí que puede decirse que la tecnología es “el arte, la técnica o la manera de hacer cosas, construir objetos y artefactos que satisfagan las necesidades de personas y comunidades, mediante la aplicación de conocimientos técnicos ordenados científicamente.”

El Diccionario de la Real Academia Española define a la tecnología como el “Conjunto de teorías y de técnicas que permiten el aprovechamiento práctico del conocimiento científico.” Así como el “Conjunto de los instrumentos y procedimientos industriales de un determinado sector o producto.”

En un mundo de vertiginosos cambios como el que estamos viviendo, la tecnología está cada vez más presente en nuestras vidas y en todo tipo de organizaciones, sean ellas empresas de producción y comercialización de bienes de consumo masivo, instituciones educativas, instituciones dedicadas al cuidado de la salud, ejércitos, empresas de transporte, agrícolas, de desarrollo de software, de provisión de servicios de comunicaciones, etc.

La sociedad de la información y de las nuevas tecnologías se ha expandido a todos los campos de la ciencia, de allí que hoy se hable de biotecnología, de nanotecnología, de tecnología informática y de comunicaciones, de tecnología educativa, etc.

Ya vivimos en la era digital en la cual el tamaño y la ubicación física están cediendo paso a la virtualidad y en la cual los robots se están haciendo cargo de muchos de los trabajos que antes tenía el ser humano. Más aún, la inteligencia artificial además les posibilita solucionar problemas que se le presentan. Los robots no solo actúan como humanos sino que también piensan por nosotros!

De hecho, viendo a la tecnología utilizada como un recurso estratégico, es frecuente que distintas organizaciones realicen prácticas de benchmarking tecnológico, comparando capacidades tecnológicas propias con las capacidades tecnológicas de la competencia.

Por eso, siendo entonces la presencia e importancia de la tecnología cada vez más profunda y abarcativa en todas las funciones de negocio y niveles decisorios, resulta oportuno analizar cómo impacta en las organizaciones actuales y anticiparse a lo que se puede esperar en el futuro.

Este es el objetivo del presente trabajo.

\section{Establezca la diferencia entre Comisión Negociadora y Comisión Paritaria.}
\label{sec:DiferenciaComisiones}
El Diccionario de la Lengua define con claridad el significado de la palabra paritaria, en los siguientes términos: ``Dicho especialmente de un organismo de carácter social: Constituido por representantes de patronos y obreros en número igual y con los mismos derechos''.

Es, pues, la estricta igualdad de número y derechos (especialmente del derecho al voto) lo que caracteriza a cualquier organismo paritario. Si esa igualdad se rompe, en cualquiera de sus extremos, simplemente el organismo deja de ser paritario.

Pero en mucho casos la participación no es igualitaria. Pero se entiende el sentido que ambas partes están en igual de condiciones de negociación.

Por otro lado, sería absurdo y contradictorio que una comisión o grupo que negocia un convenio colectivo (o cualquier otro tipo de acuerdo colectivo, incluidas las modificaciones de convenios y las actualizaciones de las escalas salariales) sea ``paritario''.

Esta posibilidad carece mayormente de sentido puesto que la negociación colectiva, como instrumento de composición del conflicto, quedaría bloqueada en un 90\% de los casos. La igualdad de voto entre los antagonistas sociales daría lugar a permanentes empates.

De acuerdo con la ley que rige los aspectos formales de la negociación colectiva en la Argentina, los convenios pueden ser negociados, acordados y firmados por ``una asociación profesional de empleadores, un empleador o un grupo de empleadores, y una asociación sindical de trabajadores con personería gremial''.

El carácter no-paritario de la mesa que negocia el convenio está perfectamente establecido en el artículo 1º de la propia ley (14.250), desde el momento en que un convenio negociado por un grupo de empleadores (es decir, varios de ellos) y un solo sindicato es perfectamente válido. Las comisiones negociadoras son mixtas, pero no paritarias.

\subsection{¿Qué son en realidad las ``paritarias''?}
La propia ley 14.250 (Modificada por la  Ley 25.877) da la respuesta: son comisiones especiales, creadas por convenio colectivo e integradas por un número igual de representantes de empleadores y trabajadores, cuyo funcionamiento y atribuciones se rigen, en principio, por las disposiciones del propio convenio, pero que, de manera general, tienen por cometidos el de intervenir en la solución de conflictos laborales derivados de la interpretación y aplicación del convenio colectivo, así como el de interpretar el convenio con carácter general y vinculante.

La ley argentina establece que las decisiones de las comisiones paritarias ``se incorporan'' al convenio colectivo, lo cual da a entender, inmediatamente, que estas comisiones no son las que dan vida al convenio.

En otras palabras, que de acuerdo con el sistema legal, es la vida de las comisiones paritarias la que depende de la existencia del convenio y no al revés.

\subsubsection{¿Qué son las comisiones negociadoras?}
La comisión negociadora de un convenio colectivo esta formada por el conjunto de personas constituido por representantes de empresarios y trabajadores y que tiene por objeto la elaboración del convenio colectivo.


\subsection{Conclusión}
La Comisión Negociadora es la que acuerda el Convenio Colectivo de Trabajo, mientras que la Comisión Paritaria entra en vigor para implementar e interpretar el Convenio.

\section{A que nos referimos cuando hablamos de:  Deber de negociar de buena fe y Derecho a la información}

Se negocia de buena fe cuando no se defrauda o abusa de la confianza del otro, cuando se guarda fidelidad a la palabra dada y cuando ambas partes son colaboradoras y solidarias. 

Los derechos que implica este principio serían: 
\begin{itemize}
\item Que ambas partes, trabajadores y empleadores, concurran a las reuniones fijadas por la autoridad de aplicación. Está como una obligación, pero puede ser entendido como un derecho por la contraparte.
\item Para que una negociación sea transparente, y de buena fe, la información a intercambiar, y necesaria, debe ser verídica y suficiente para la negociación. Dicho intercambio deberá obligatoriamente incluir la información relativa a la distribución de los beneficios de la productividad, la situación actual del empleo y las previsiones sobre su futura evolución.
\item Este información, de ambas partes, debe ser suficiente para la negociación más no completa. Nadie está obligado a brindar información que no sea requerida para solucionar el conflicto.
\end{itemize}

Esto implica que cualquier parte tiene derecho a solicitar información que de la negociación resulte necesaria para explicar fundamentos de las posiciones (Sacado del Manual de Recursos Humanos, del Postgrado de Recursos Humanos, sin autor y título)

\section{Atribuciones de las Comisiones Paritarias.}
De acuerdo al texto proporcionado por la cátedra se enumera las atribuciones. El siguiente es un breve resumen:

\begin{itemize}
\item \textbf{Funciones Interpretativas}: regla los alcances de lo acordado en un Convenio Colectivo de Trabajo ante un conflicto laboral concreto en cuanto a su formulación. Pero su interpretación es general. Es decir que interpretado del Convenio es de carácter general para todos los trabajadores y para todos los casos en el tiempo futuro.

\item \textbf{Funciones Conciliatorias}: De surgir conflictos de orden laboral, esta comisión puede entablar conversaciones para buscar soluciones en orden de buscar una solución negociada. En este caso el conflicto se presenta como casos particulares de problemas de trabajadores individuales o grupos de trabajadores. Este rol tiene que ver con prácticas organizacionales y la administración de los recursos humanos bajo las pautas del Convenio y lo acordado en el anterior rol en su función interpretativa. 

\item \textbf{Funciones Normativas}: Mas allá de la declaración ideológica del texto en que menciona un rumbo hacia un modelo post Taylorismo, es el rol más importante para pues en esta rol la comisión puede recrear los aspectos básicos de los principios del Convenio Colectivo adecuándolo a la realidad organizativa donde tiene incumbencia, teniendo en cuanta los usos y costumbres, los avances técnicos-científicos y la evolución organizativa.

\item \textbf{Funciones Complementarias}: Un rol muy vago e inespecífico. Propio de lo que se quiere encausar es las relaciones laborales, y en un modo más general relaciones humanas en un marco económico-productivo con diferentes intereses. Desde este rol es donde se puede tratar problemáticas que suceden en el contexto cercano o interno, en materia laboral o acerca de la propia actividad organizacional donde funciona la Comisión. 

\end{itemize}



\section{Sobre el Trabajo}
Para este trabajo se utilizó la planilla "Quantum Hall effect report", recomendada para "Homework Assignment" de la Universidad de Copenhague.

El código fuente del documento se encuentra en:

https://www.overleaf.com/read/vknzvkjbdbbk

\end{document}



