\documentclass[a4paper, 12pt]{article}

\usepackage[spanish]{babel}
\usepackage[utf8]{inputenc}
\usepackage{graphicx}
\usepackage[colorinlistoftodos]{todonotes}

\title{Descripciôn teorica de la relaciôn Tecnología y las organizaciones}

\author{Carlos M. Dagorret}

\date{\today}

\begin{document}
\maketitle

\begin{abstract}
En un mundo de vertiginosos cambios como el que estamos viviendo, la tecnología está cada vez más presente en nuestras vidas y en todo tipo de organizaciones, sean ellas empresas de producción y comercialización de bienes de consumo masivo, instituciones educativas, instituciones dedicadas al cuidado de la salud, ejércitos, empresas de transporte, agrícolas, de desarrollo de software, de provisión de servicios de comunicaciones, etc.

La sociedad de la información y de las nuevas tecnologías se ha expandido a todos los campos de la ciencia, de allí que hoy se hable de biotecnología, de nanotecnología, de tecnología informática y de comunicaciones, de tecnología educativa, etc.

Ya vivimos en la era digital en la cual el tamaño y la ubicación física están cediendo paso a la virtualidad y en la cual los robots se están haciendo cargo de muchos de los trabajos que antes tenía el ser humano. Más aún, la inteligencia artificial además les posibilita solucionar problemas que se le presentan. Los robots no solo actúan como humanos sino que también piensan por nosotros.

El presente trabajo tiene un fin descriptivo de los elementos en conflicto en una organización y la tecnología de la información.
\end{abstract}

\section{Introducción}
\label{sec:Introduccion}

La palabra tecnología viene del griego ''tekhne'' (arte, técnica u oficio) y logos (estudio, tratado). De allí que puede decirse que la tecnología es ''el arte, la técnica o la manera de hacer cosas, construir objetos y artefactos que satisfagan las necesidades de personas y comunidades, mediante la aplicación de conocimientos técnicos ordenados científicamente.''

El Diccionario de la Real Academia Española define a la tecnología como el ''Conjunto de teorías y de técnicas que permiten el aprovechamiento práctico del conocimiento científico.'' Así como el ''Conjunto de los instrumentos y procedimientos industriales de un determinado sector o producto.''

En un mundo de vertiginosos cambios como el que estamos viviendo, la tecnología está cada vez más presente en nuestras vidas y en todo tipo de organizaciones, sean ellas empresas de producción y comercialización de bienes de consumo masivo, instituciones educativas, instituciones dedicadas al cuidado de la salud, ejércitos, empresas de transporte, agrícolas, de desarrollo de software, de provisión de servicios de comunicaciones, etc.

La sociedad de la información y de las nuevas tecnologías se ha expandido a todos los campos de la ciencia, de allí que hoy se hable de biotecnología, de nanotecnología, de tecnología informática y de comunicaciones, de tecnología educativa, etc.

Ya vivimos en la era digital en la cual el tamaño y la ubicación física están cediendo paso a la virtualidad y en la cual los robots se están haciendo cargo de muchos de los trabajos que antes tenía el ser humano. Más aún, la inteligencia artificial además les posibilita solucionar problemas que se le presentan. Los robots no solo actúan como humanos sino que también piensan por nosotros.

De hecho, viendo a la tecnología utilizada como un recurso estratégico, es frecuente que distintas organizaciones realicen prácticas de benchmarking tecnológico, comparando capacidades tecnológicas propias con las capacidades tecnológicas de la competencia.

Por eso, siendo entonces la presencia e importancia de la tecnología cada vez más profunda y abarcativa en todas las funciones de negocio y niveles decisorios, resulta oportuno analizar cómo impacta en las organizaciones actuales y anticiparse a lo que se puede esperar en el futuro.

Este es el objetivo del presente trabajo.

\section{Antecedentes}
\label{sec:Antecedentes}
Desde siempre, las revoluciones industriales y muchos de los grandes avances de la humanidad se caracterizaron, y tuvieron su origen, en profundas transformaciones tecnológicas. A su vez, estas transformaciones, realimentaron el sistema y potenciaron el desarrollo tecnológico de cada momento.

Ya hace más de 30 años, John Naisbitt en su libro Megatendencias \cite{Naisbitt1984} al referirse a las nuevas orientaciones que están transformando nuestra vida, señalaba que, de la sociedad industrial a la que tanto aportaron los autores clásicos de la administración, se estaba pasando a la sociedad de la información y que de la tecnología aplicada en apoyo de las tradicionales líneas de producción se estaba conformando una fábrica basada en computadores y robots.

También Alvin Toffler\cite{Toffler1997} en su obra La tercera ola se refiere a la revolución tecnológica y de la información propia de nuestros tiempos y que nos lleva a vivir en una sociedad post-industrial.

Según Toffler, caracterizan a la tercera ola la desarticulación de estructuras de la segunda ola, a saber:
\begin{itemize}
\item Descentralización
\item Desmasificación
\item Personalización
\end{itemize}

La producción actual en serie es complementada con la producción en series cortas. La producción ya no se dedica a hacer decenas de miles de ejemplares de un único producto, sino cientos de ejemplares de cientos de productos. Esto depende la rama de actividad y la flexibilidad del sistema de producción. Hoy,  a fines de la primera década del segundo milenio, las impresoras 3D son el ejemplo de mayor flexibilidad productiva.

Así encontramos productos cada vez más personalizados.

En la actualidad fundamentalmente se amplifica la fuerza mental del ser humano. Los sistemas cibernéticos, computacionales, de comunicación e Internet, funcionan como amplificadores de esa fuerza mental.

Si bien las nuevas generaciones asimilan de modo natural la nueva cultura que se va conformando, para muchos de nosotros conlleva frecuentemente realizar importantes esfuerzos de asimilación y adaptación descongelando los viejos paradigmas.

Los avances tecnológicos y el desarrollo de las telecomunicaciones contribuyeron a diseñar una nueva sociedad ya que crearon una infraestructura favorable para el surgimiento de un fenómeno histórico trascendental:  la globalización, que no es más que un proceso de expansión a nivel mundial de actividades humanas importantes como las económicas, políticas y culturales.

Hoy, al igual que la producción, los medios se van desmasificando.

La tecnología ha desafiado no solo al tiempo y a la distancia sino también los fundamentos científicos sobre la mente y la inteligencia. Una computadora puede ser considerada mucho más que como un objeto de guarda de datos, generación de información y de cálculo, sino como un verdadero objeto cultural.

Dentro de esos cambios tecnológicos adquieren particular relevancia los cambios producidos en materia de tecnología informática.  La relevancia nos muestra el ranking del top 10 de las compañías:

\begin{table}[]
\begin{tabular}{lll}
Ranking & Empresa               & \begin{tabular}[c]{@{}l@{}}Varlor en Billones\\ de U\$S\end{tabular} \\
1       & Apple                 & 833,25                                                         \\
2       & Amazon                & 734,85                                                         \\
3       & Microsoft             & 725,78                                                         \\
4       & Alphabet              & 723,48                                                         \\
5       & Facebook              & 505,93                                                         \\
6       & Berkshire Hathaway    & 489,0                                                          \\
7       & Alibaba               & 439,85                                                         \\
8       & JP Morgan Chase \& Co & 377,85                                                         \\
9       & Johnson \& Johnson    & 343,43                                                         \\
10      & Exxon Mobil           & 342,64                                                        
\end{tabular}
\caption{Valor de Mercado Promedio últimos 12 meses. Abril, 2019}
\label{table:1}
\end{table}


Tal es así que Stan Davis y Bill Davidson, citados por Tom Peters en su libro la Gerencia Liberada4, señalan “Hacia el año 2020, el 80 % de las ganancias comerciales y del valor del mercado provendrán de ese sector de la empresa que está construida alrededor de los negocios de la información.”

Gibson5, al referirse a la eficacia de las organizaciones señala que la tecnología, junto con las opciones estratégicas, la estructura organizacional, los procesos y la cultura, son causas determinantes del éxito organizacional. Además la capacidad técnica y los conocimientos de los individuos, o sea la capacidad para aprovechar la tecnología disponible es causa determinante de la eficacia a nivel personal.

Los avances tecnológicos son poderosas fuerzas del cambio que, si no se entienden o no se aplican, aseguran la pérdida de eficacia organizacional en un plazo cada vez más corto y, en consecuencia, ponen en peligro la supervivencia de la organización.

Hasta la segunda mitad del siglo pasado los equipos y bienes de consumo se consideraban algo que debía durar ya que se invertía mucho dinero en ellos y debían estar en uso tanto tiempo como fuera posible.

En los años XX del siglo pasado se empezó a concebir lo que conocemos en la actualidad como la obsolescencia programada, esto es el intento por parte del fabricante de un bien o servicio de reducir su ciclo de vida para que el consumidor se vea obligado a adquirir otro similar

Vance Packard en su libro The Waste Makers6 señala 3 tipos de obsolescencia

Obsolescencia de función: Se da cuando un producto sustituye a otro por su funcionalidad superior.

Obsolescencia de calidad: Se da cuando el producto se vuelve obsoleto por un mal funcionamiento programado.

Obsolescencia de deseo: Ocurre cuando el producto, aun siendo completamente funcional y no habiendo sustituto mejor, deja de ser deseado por cuestiones de moda o estilo, y se le asignan valores peyorativos que disminuyen su deseo de compra y animan a su sustitución.

Hoy la obsolescencia programada, especialmente aquélla que se relaciona con la función y el deseo, muestra un ciclo cada vez más corto ya que el avance tecnológico


\subsubsection{¿Qué son las comisiones negociadoras?}
La comisión negociadora de un convenio colectivo esta formada por el conjunto de personas constituido por representantes de empresarios y trabajadores y que tiene por objeto la elaboración del convenio colectivo.


\subsection{Conclusión}
La Comisión Negociadora es la que acuerda el Convenio Colectivo de Trabajo, mientras que la Comisión Paritaria entra en vigor para implementar e interpretar el Convenio.

\section{A que nos referimos cuando hablamos de:  Deber de negociar de buena fe y Derecho a la información}

Se negocia de buena fe cuando no se defrauda o abusa de la confianza del otro, cuando se guarda fidelidad a la palabra dada y cuando ambas partes son colaboradoras y solidarias. 

Los derechos que implica este principio serían: 
\begin{itemize}
\item Que ambas partes, trabajadores y empleadores, concurran a las reuniones fijadas por la autoridad de aplicación. Está como una obligación, pero puede ser entendido como un derecho por la contraparte.
\item Para que una negociación sea transparente, y de buena fe, la información a intercambiar, y necesaria, debe ser verídica y suficiente para la negociación. Dicho intercambio deberá obligatoriamente incluir la información relativa a la distribución de los beneficios de la productividad, la situación actual del empleo y las previsiones sobre su futura evolución.
\item Este información, de ambas partes, debe ser suficiente para la negociación más no completa. Nadie está obligado a brindar información que no sea requerida para solucionar el conflicto.
\end{itemize}

Esto implica que cualquier parte tiene derecho a solicitar información que de la negociación resulte necesaria para explicar fundamentos de las posiciones (Sacado del Manual de Recursos Humanos, del Postgrado de Recursos Humanos, sin autor y título)

\section{Atribuciones de las Comisiones Paritarias.}
De acuerdo al texto proporcionado por la cátedra se enumera las atribuciones. El siguiente es un breve resumen:

\begin{itemize}
\item \textbf{Funciones Interpretativas}: regla los alcances de lo acordado en un Convenio Colectivo de Trabajo ante un conflicto laboral concreto en cuanto a su formulación. Pero su interpretación es general. Es decir que interpretado del Convenio es de carácter general para todos los trabajadores y para todos los casos en el tiempo futuro.

\item \textbf{Funciones Conciliatorias}: De surgir conflictos de orden laboral, esta comisión puede entablar conversaciones para buscar soluciones en orden de buscar una solución negociada. En este caso el conflicto se presenta como casos particulares de problemas de trabajadores individuales o grupos de trabajadores. Este rol tiene que ver con prácticas organizacionales y la administración de los recursos humanos bajo las pautas del Convenio y lo acordado en el anterior rol en su función interpretativa. 

\item \textbf{Funciones Normativas}: Mas allá de la declaración ideológica del texto en que menciona un rumbo hacia un modelo post Taylorismo, es el rol más importante para pues en esta rol la comisión puede recrear los aspectos básicos de los principios del Convenio Colectivo adecuándolo a la realidad organizativa donde tiene incumbencia, teniendo en cuanta los usos y costumbres, los avances técnicos-científicos y la evolución organizativa.

\item \textbf{Funciones Complementarias}: Un rol muy vago e inespecífico. Propio de lo que se quiere encausar es las relaciones laborales, y en un modo más general relaciones humanas en un marco económico-productivo con diferentes intereses. Desde este rol es donde se puede tratar problemáticas que suceden en el contexto cercano o interno, en materia laboral o acerca de la propia actividad organizacional donde funciona la Comisión. 

\end{itemize}



\section{Sobre el Trabajo}
Para este trabajo se utilizó la planilla "Quantum Hall effect report", recomendada para "Homework Assignment" de la Universidad de Copenhague.

El código fuente del documento se encuentra en:

https://www.overleaf.com/read/vknzvkjbdbbk

\begin{thebibliography}{9}
\bibitem{Naisbitt1984}
Naisbitt, J. \emph{Megatendencias Dieznuevas direcciones de cambio..}. 1984 Buenos Aires, Fundación CERIEN.

\bibitem{Toffler1997}
Toffler, \emph{A. La tercera ola}. 1997. Barcelona, Plaza y Janes. 
\end{thebibliography}

\end{document}



