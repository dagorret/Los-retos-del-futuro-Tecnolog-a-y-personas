\documentclass[a4paper, 12pt]{article}

\usepackage[spanish]{babel}
\usepackage[utf8]{inputenc}
\usepackage{graphicx}
\usepackage[colorinlistoftodos]{todonotes}

\title{Descripciôn teorica de la relaciôn Tecnología y las organizaciones}

\author{Carlos M. Dagorret}

\date{\today}

\begin{document}
\maketitle

\begin{abstract}
En un mundo de vertiginosos cambios como el que estamos viviendo, la tecnología está cada vez más presente en nuestras vidas y en todo tipo de organizaciones, sean ellas empresas de producción y comercialización de bienes de consumo masivo, instituciones educativas, instituciones dedicadas al cuidado de la salud, ejércitos, empresas de transporte, agrícolas, de desarrollo de software, de provisión de servicios de comunicaciones, etc.

La sociedad de la información y de las nuevas tecnologías se ha expandido a todos los campos de la ciencia, de allí que hoy se hable de biotecnología, de nanotecnología, de tecnología informática y de comunicaciones, de tecnología educativa, etc.

Ya vivimos en la era digital en la cual el tamaño y la ubicación física están cediendo paso a la virtualidad y en la cual los robots se están haciendo cargo de muchos de los trabajos que antes tenía el ser humano. Más aún, la inteligencia artificial además les posibilita solucionar problemas que se le presentan. Los robots no solo actúan como humanos sino que también piensan por nosotros.

El presente trabajo tiene un fin descriptivo de los elementos en conflicto en una organización y la tecnología de la información.
\end{abstract}

\section{Introducción}
\label{sec:Introduccion}

La palabra tecnología viene del griego ''tekhne'' (arte, técnica u oficio) y logos (estudio, tratado). De allí que puede decirse que la tecnología es ''el arte, la técnica o la manera de hacer cosas, construir objetos y artefactos que satisfagan las necesidades de personas y comunidades, mediante la aplicación de conocimientos técnicos ordenados científicamente.''

El Diccionario de la Real Academia Española define a la tecnología como el ''Conjunto de teorías y de técnicas que permiten el aprovechamiento práctico del conocimiento científico.'' Así como el ''Conjunto de los instrumentos y procedimientos industriales de un determinado sector o producto.''

En un mundo de vertiginosos cambios como el que estamos viviendo, la tecnología está cada vez más presente en nuestras vidas y en todo tipo de organizaciones, sean ellas empresas de producción y comercialización de bienes de consumo masivo, instituciones educativas, instituciones dedicadas al cuidado de la salud, ejércitos, empresas de transporte, agrícolas, de desarrollo de software, de provisión de servicios de comunicaciones, etc.

La sociedad de la información y de las nuevas tecnologías se ha expandido a todos los campos de la ciencia, de allí que hoy se hable de biotecnología, de nanotecnología, de tecnología informática y de comunicaciones, de tecnología educativa, etc.

Ya vivimos en la era digital en la cual el tamaño y la ubicación física están cediendo paso a la virtualidad y en la cual los robots se están haciendo cargo de muchos de los trabajos que antes tenía el ser humano. Más aún, la inteligencia artificial además les posibilita solucionar problemas que se le presentan. Los robots no solo actúan como humanos sino que también piensan por nosotros.

De hecho, viendo a la tecnología utilizada como un recurso estratégico, es frecuente que distintas organizaciones realicen prácticas de benchmarking tecnológico, comparando capacidades tecnológicas propias con las capacidades tecnológicas de la competencia.

Por eso, siendo entonces la presencia e importancia de la tecnología cada vez más profunda y abarcativa en todas las funciones de negocio y niveles decisorios, resulta oportuno analizar cómo impacta en las organizaciones actuales y anticiparse a lo que se puede esperar en el futuro.

Este es el objetivo del presente trabajo.

\section{Antecedentes}
\label{sec:Antecedentes}
Desde siempre, las revoluciones industriales y muchos de los grandes avances de la humanidad se caracterizaron, y tuvieron su origen, en profundas transformaciones tecnológicas. A su vez, estas transformaciones, realimentaron el sistema y potenciaron el desarrollo tecnológico de cada momento.

Ya hace más de 30 años, John Naisbitt en su libro Megatendencias \cite{Naisbitt1984} al referirse a las nuevas orientaciones que están transformando nuestra vida, señalaba que, de la sociedad industrial a la que tanto aportaron los autores clásicos de la administración, se estaba pasando a la sociedad de la información y que de la tecnología aplicada en apoyo de las tradicionales líneas de producción se estaba conformando una fábrica basada en computadores y robots.

También Alvin Toffler\cite{Toffler1997} en su obra La tercera ola se refiere a la revolución tecnológica y de la información propia de nuestros tiempos y que nos lleva a vivir en una sociedad post-industrial.

Según Toffler, caracterizan a la tercera ola la desarticulación de estructuras de la segunda ola, a saber:
\begin{itemize}
\item Descentralización
\item Desmasificación
\item Personalización
\end{itemize}

La producción actual en serie es complementada con la producción en series cortas. La producción ya no se dedica a hacer decenas de miles de ejemplares de un único producto, sino cientos de ejemplares de cientos de productos. Esto depende la rama de actividad y la flexibilidad del sistema de producción. Hoy,  a fines de la primera década del segundo milenio, las impresoras 3D son el ejemplo de mayor flexibilidad productiva.

Así encontramos productos cada vez más personalizados.

En la actualidad fundamentalmente se amplifica la fuerza mental del ser humano. Los sistemas cibernéticos, computacionales, de comunicación e Internet, funcionan como amplificadores de esa fuerza mental.

Si bien las nuevas generaciones asimilan de modo natural la nueva cultura que se va conformando, para muchos de nosotros conlleva frecuentemente realizar importantes esfuerzos de asimilación y adaptación descongelando los viejos paradigmas.

Los avances tecnológicos y el desarrollo de las telecomunicaciones contribuyeron a diseñar una nueva sociedad ya que crearon una infraestructura favorable para el surgimiento de un fenómeno histórico trascendental:  la globalización, que no es más que un proceso de expansión a nivel mundial de actividades humanas importantes como las económicas, políticas y culturales.

Hoy, al igual que la producción, los medios se van desmasificando.

La tecnología ha desafiado no solo al tiempo y a la distancia sino también los fundamentos científicos sobre la mente y la inteligencia. Una computadora puede ser considerada mucho más que como un objeto de guarda de datos, generación de información y de cálculo, sino como un verdadero objeto cultural.

Dentro de esos cambios tecnológicos adquieren particular relevancia los cambios producidos en materia de tecnología informática.  La relevancia nos muestra el ranking del top 10 de las compañías:

\begin{table}[]
\begin{tabular}{lll}
Ranking & Empresa               & \begin{tabular}[c]{@{}l@{}}Varlor en Billones\\ de U\$S\end{tabular} \\
1       & Apple                 & 833,25                                                         \\
2       & Amazon                & 734,85                                                         \\
3       & Microsoft             & 725,78                                                         \\
4       & Alphabet              & 723,48                                                         \\
5       & Facebook              & 505,93                                                         \\
6       & Berkshire Hathaway    & 489,0                                                          \\
7       & Alibaba               & 439,85                                                         \\
8       & JP Morgan Chase \& Co & 377,85                                                         \\
9       & Johnson \& Johnson    & 343,43                                                         \\
10      & Exxon Mobil           & 342,64                                                        
\end{tabular}
\caption{Valor de Mercado Promedio últimos 12 meses. Abril, 2019}
\label{table:1}
\end{table}



En las organizaciones la tecnología, junto con las opciones estratégicas, la estructura organizacional, los procesos y la cultura, son causas determinantes del éxito organizacional. 

Además la capacidad técnica y los conocimientos de los individuos, o sea la capacidad para aprovechar la tecnología disponible es causa determinante de la eficacia a nivel personal.

Los avances tecnológicos son poderosas fuerzas del cambio que, si no se entienden o no se aplican, aseguran la pérdida de eficacia organizacional en un plazo cada vez más corto y, en consecuencia, ponen en peligro la supervivencia de la organización.

Hasta la segunda mitad del siglo pasado los equipos y bienes de consumo se consideraban algo que debía durar ya que se invertía mucho dinero en ellos y debían estar en uso tanto tiempo como fuera posible.

En los últimos decenios del siglo XX se empezó a concebir lo que conocemos en la actualidad como la obsolescencia programada, esto es el intento por parte del fabricante de un bien o servicio de reducir su ciclo de vida para que el consumidor se vea obligado a adquirir otro similar

Vance Packard{\cite{Packard1964}} en su libro The Waste Makers6 señala 3 tipos de obsolescencias:

\begin{enumerate}
\item Obsolescencia de función: Se da cuando un producto sustituye a otro por su funcionalidad superior.
\item Obsolescencia de calidad: Se da cuando el producto se vuelve obsoleto por un mal funcionamiento programado.
\item Obsolescencia de deseo: Ocurre cuando el producto, aun siendo completamente funcional y no habiendo sustituto mejor, deja de ser deseado por cuestiones de moda o estilo, y se le asignan valores peyorativos que disminuyen su deseo de compra y animan a su sustitución.
\end{enumerate}

Hoy la obsolescencia programada, especialmente aquélla que se relaciona con la función y el deseo, muestra un ciclo cada vez más corto ya que el avance tecnológico no se detiene, al contrario, se multiplica a una tasa exponencial ya que se realimenta continuamente y a mayor velocidad.

Nuevos productos, nuevos servicios y nuevos métodos están presentes cada día de nuestras vidas.

La evolución tecnológica no solo es vertiginosa, sino también disruptiva, ejemplo de ello fue la aparición de las computadoras personales que modificaron sustancialmente la forma de producir, tomar decisiones y hacer negocios , o los teléfonos celulares que cambiaron totalmente los modelos de comunicación a nivel personal y empresario.

\section{El desarrollo tecnológico actual}
\label{sec:DesarrolloTecnologicoActual}
Si bien como se ha señalado, la evolución tecnológica abarca a todas las áreas del saber humano, es frecuentemente compartido que entre las nuevas tecnologías de propósito general que se destacan en la actualidad se encuentran las biotecnologías, las de nuevos materiales, las energéticas (comprendiendo la búsqueda de energías limpias), la robótica y las tecnologías de la información y comunicación (TICs).

En particular y, teniendo en cuenta los objetivos del presente trabajo, si bien hare referencia a todas estas tecnologías, el foco se concentrará, atento al impacto que tienen en las organizaciones, en las tecnologías relacionadas con las TICs.

\subsection{TICs}
\label{TICS}
En el área de tecnología informática hoy se están produciendo tres cambios fundamentales que están interrelacionados.

\begin{itemize}
\item \textbf{Plataformas digitales móviles.} Smartphones y tablets acercan la información donde se encuentra el individuo y la proporcionan en el momento en que la necesita. Esto produce mejoras productivas y obra como diferenciador competitivo. Es importante que las plataformas móviles estén debidamente integradas con los procesos centrales.

\item \textbf{Crecimiento del Software en línea como un servicio.}
Este es un modelo de distribución de software donde el soporte lógico y los datos que maneja se alojan en servidores de empresas proveedoras de tecnologías de información y comunicación, a las que se accede vía Internet. Estas empresas se ocupan del servicio de mantenimiento, de la operación diaria y del soporte del software usado por el cliente. La información, el procesamiento, los insumos, y los resultados de la lógica de negocio del software, están hospedados en la compañía del proveedor.

\item \textbf{Crecimiento de la computación en la nube.}
Este es un modelo que provee acceso a una reserva compartida de recursos computacionales (computadores, almacenamiento, aplicaciones y servicios). 
Se reduce así la necesidad de contar con hardware y software propio, con los consiguientes ahorros y con el aprovechamiento de las experiencias de los proveedores de estos servicios.

\end{itemize}

Las empresas que ofrecen dichos servicios a nivel global están, no causalmente, en el Top 5 del Cuadro 1.

Estas  cuentan con una infraestructura de servicios que permite que las personas puedan crear aplicaciones que escalen a cientos de miles de usuarios a un costo relativamente bajo y sin preocuparse por el mantenimiento y disponibilidad de los servidores.

\subsubsection{Teleinformática}
\label{TeleInformatica}


La teleinformática es la ciencia que trata la conectividad y comunicación a distancia entre procesos; es la rama que estudia la transmisión y comunicación de información mediante vía de equipos informáticos.

La expansión de la teleinformática en los últimos años se vio fuertemente potenciada a partir del uso generalizado de Internet.

Hoy existe un mercado digital en el cual millones de personas de todo el mundo pueden intercambiar cantidades masivas de información en forma directa, al instante y a un costo insignificante.

Ya sea a partir de la utilización de Internet como de otros medios de comunicación se han desarrollados distintos tipos de negocio electrónico:

\begin{itemize}
    \item \textbf{Negocio al consumidor (B2C)}:  Venta al detalle de productos y servicios a compradores individuales.
    
    \item \textbf{Negocio a negocio (B2B)} :Venta de productos y servicios entre empresas.
    
    \item \textbf{Negocio de consumidor a consumidor (C2C)}
\end{itemize}


\subsubsection{Clasificación de	sistemas de información según el nivel de la jerarquía organizacional}
\label{ClasificacionJerarquia}

Toda plataforma de tecnología informática comprende una serie de sistemas que incluye software empresarial, sistemas operativos, bases de datos, plataformas de internet. etc.

Dentro de lo que se conoce como software empresarial, existen distintos tipos de sistemas de información.

Estos sistemas pueden clasificarse de distintas formas, Una de esas formas consiste en relacionarlos con el nivel de la jerarquía organizacional que más los utiliza:

\paragraph{Sistemas de procesamiento de transacciones TPS (Transaction Processing Systems)}

Son sistemas dedicados al procesamiento de transacciones de toda índole tales como son, por ejemplo la generación de facturas, emisión de recibos de sueldos, generación de órdenes de producción, información de inventarios, generación de órdenes de pago, etc. En general estos son los sistemas que más se utilizan en el llamado núcleo operativo de las organizaciones.

Por tratarse de sistemas básicos para la realización de las actividades operativas primarias suelen ser los primeros sistemas que se implementan. 

Si bien manejan datos en forma intensiva, sus cálculos y procesos suelen ser simples y nada sofisticados. 

\paragraph{Sistemas de información gerencial MIS (Management Information Systems)}

Son sistemas orientados a solucionar problemas empresariales en general, por lo cual son comúnmente utilizados por los niveles gerenciales medios y altos.

Estos son los sistemas incluidos en el denominado tablero de comando o tablero de control ya que a partir de los datos que provienen fundamentalmente de los sistemas de procesamiento transaccionales permiten la generación de información en forma flexible según las necesidades que se tengan para la toma de decisiones.

Habitualmente tratan problemas estructurados y semiestructurados. Informes por excepción y señales de alarma están presentes en este tipo de sistemas.

\paragraph{Sistemas de soporte de decisiones DSS (Decision Support Systems)}

Son herramientas útiles para realizar el análisis de las diferentes variables de negocio con la finalidad de apoyar el proceso de toma de decisiones.

Estos son sistemas informáticos interactivos, que permiten extraer y manipular información de manera flexible y que ayudan a quienes deben tomar decisiones utilizando datos y modelos a resolver problemas no estructurados.

Tienen amplias posibilidades para la elaboración de pronósticos, la evaluación, simulación y/o la comparación de alternativas y análisis de sensibilidad. Las planillas de cálculo son frecuentemente utilizadas con este propósito ya que responden a la pregunta ¿Qué pasa si? 

Son sistemas utilizados en forma frecuente por las gerencias funcionales, comprendiendo a la tecno-estructura organizacional y niveles medios de las organizaciones.

Entre los múltiples temas que suelen incluirse en estos sistemas se encuentran los de análisis de costos, análisis y fijación de precios, programación de la producción, programación financiera, etc.

\paragraph{Sistemas de inteligencia de negocios BIS (Business Information Systems)}

Estos sistemas permiten manejar aspectos estratégicos y tendencias a largo plazo. 

Tienen la particularidad de conjugar las informaciones que provienen de los niveles inferiores de la organización a través de sus sistemas transaccionales con aquellas que provienen desde fuera de ella y que permiten evaluar su posición competitiva. 

Son sistemas utilizados fundamentalmente por la cumbre estratégica y por las gerencias del más alto nivel.

El cuadro de mando integral o Balanced Scorecard puede ser diseñado, generado y analizado a partir de la información que proporciona este tipo de sistemas.

Detrás de este concepto se encuentra el de “Big Data”, datos masivos o a gran escala que consiste en la acumulación masiva de datos y a los procedimientos usados para identificar patrones recurrentes dentro de esos datos. Dentro de esos procedimientos de análisis de datos se destaca el “Data Mining” o minería de datos ya que su objetivo es el de encontrar comportamientos predictivos.

\subsubsection{Clasificación de sistemas de información según su función}
\label{ClasificacionFuncion}

\paragraph{Sistemas de automatización de oficinas (OAS: Office Automation Systems)}

Son aplicaciones destinadas a ayudar al trabajo administrativo diario de una organización. 

Entre los componentes más comunes de un OAS están el procesamiento de texto, las hojas de cálculo, la autoedición, la calendarización electrónica y las comunicaciones mediante correo de voz, correo electrónico y videoconferencias.

Sin embargo, dependiendo de la industria en la que se trabaje, el tradicional correo electrónico está quedando obsoleto ya que se buscan nuevos sistemas de comunicación y en tiempo real.

Así, se formaliza el uso de herramientas de chat, como Flowdock o Slack que buscan mejorar la forma en la que la gente se comunica.

En cuanto al procesamiento de texto y hojas de cálculo sucede lo mismo ya que se está empezando a utilizar Google Docs u Office Online, que con la misma funcionalidad que softwares tradicionales, permite que los documentos estén en la nube, sean accesibles y visibles para todos los usuarios, permitiendo el trabajo cooperativo o simultaneo,  con los que se comparten, evitando que al pasar documentos por correo electrónico se pierda información o no se trabaje con la versiones más actualizadas.

\paragraph{Sistemas de planificación de recursos empresariales (ERP: Enterprise Resource Planning)}

Los sistemas ERP típicamente manejan en forma integrada y en tiempo real, entre muchas otras, las operaciones de producción, logística, distribución, inventario, envíos, facturación, gestión de recursos humanos, contabilidad y finanzas de la compañía de forma modular. 

En su operación interactúan gran parte de los departamentos de una organización. Su base es la de los sistemas transaccionales ya comentados.

Este tipo de sistemas permite consolidar los datos de negocio de modo de no duplicar esfuerzos, así como optimizar los procesos empresariales y responder de manera rápida a las demandas de clientes.

Estos sistemas están debidamente integrados a dispositivos móviles e modo de incrementar la eficiencia de sus usuarios.

\paragraph{Sistemas   administración   de   redes   de suministro (SCM: Supply Chain Management)}
Estos sistemas que frecuentemente forman parte de los sistemas ERP mencionados y permiten extender la gestión más allá de los límites de la empresa. 

Su operatoria enfoca al proceso de planificación, puesta en ejecución y control de las operaciones de la red de suministro que conforma la cadena de valor con el propósito de satisfacer las necesidades del cliente con tanta eficacia como sea posible, agregando valor y minimizando costos.

La gestión de la cadena de suministro atraviesa todo el movimiento y almacenaje de materias primas, productos en proceso y terminados, su correspondiente inventario y el transporte a lo largo de todo el proceso, desde la extracción u obtención de las materias primas hasta la venta y distribución de los productos terminados.

La correcta administración de la cadena de suministro debe considerar todos los acontecimientos y factores posibles que puedan causar una interrupción.

\paragraph{Sistemas de gestión de relaciones con el cliente (CRM: Customer Relationship Management)}

Son sistemas informáticos de apoyo a la gestión de las relaciones con los clientes, a la venta y al marketing.

Estos sistemas, que también pueden formar parte de sistemas ERP, comprenden varias funcionalidades para gestionar las ventas y los clientes de la empresa: automatización y promoción de ventas, tecnologías “data warehouse” (almacén de datos) e indicadores claves de negocio, funcionalidades para seguimiento de campañas de marketing y gestión de oportunidades de negocio, capacidades predictivas y de proyección de ventas.

Los beneficios del CRM no sólo se concretan en la retención y la lealtad de los clientes, a partir de un conocimiento más completo de los mismos, sino también en tener un marketing más efectivo, crear inteligentes oportunidades de “cross-selling” y abrir la posibilidad a una rápida introducción de nuevos productos o marcas.

En definitiva, lo que buscan las empresas es reducir el costo de obtener nuevos clientes, potenciar las ventas a los clientes actuales y, a la vez, incrementar la lealtad de los que ya existen ya que, como se dice habitualmente, forman parte de uno de los activos más valiosos de la empresa.

\paragraph{Sistemas de gestión del conocimiento (KMS:
Knowledge Management Systems)}

Este es un concepto aplicado en las organizaciones el cual se trata de "que cada uno en la empresa sepa lo que el otro conoce con el objeto de mejorar los resultados del negocio".

Tiene el fin de promover la creación de conocimiento, transferirlo desde el lugar dónde se genera hasta el lugar en dónde se va a aplicar e implica el desarrollo de las competencias necesarias al interior de las organizaciones para compartirlo y utilizarlo entre sus miembros, así como para valorarlo y asimilarlo si se encuentra en el exterior de éstas.

Estos sistemas deben garantizar que el nuevo conocimiento y la experiencia técnica se integren adecuadamente en la organización, evitando que el conocimiento se pierda cuando los individuos dejan la organización. El “conocimiento individual” se debe convertir en “conocimiento organizacional” que perdure en el tiempo.

Estos sistemas son habitualmente módulos de los sistemas ERP avanzados.

\paragraph{Sistemas para mejorar la colaboración y el trabajo en equipo}
Más allá de las facilidades que hoy se tienen para la realización de teleconferencias, chats , uso de Skype, etc., existen una serie de aplicaciones que sirven para mejorar la cooperación tanto intradepartamental como interdepartamental.

Entre esas se encuentra las facilidades que proporcionan las herramientas de correo electrónico, las redes Intranet, la definiciones de carpetas compartidas, o las facilidades que proporcionan aplicaciones tales como videoconferencias ad hoc.

En este tipo de sistemas también se pueden incluir los sistemas utilizados para la gestión de proyectos, los cuales permiten formular proyectos utilizando métodos Gantt, PERT o CPM y realizar la actualización y seguimiento de los mismos en forma colaborativa entre todos los integrantes de un proyecto.

\section{El impacto de la tecnología en las organizaciones}
\subsection{En la cultura organizacional}

El impacto de la tecnología en la cultura de las organizaciones es la base de las teorías de los sistemas socio técnicos. 

En estos sistemas se da una continua interacción entre los grupos humanos y la tecnología que utilizan las organizaciones. 

Es sabido que cambios en el sistema técnico deben necesariamente llevar a cambios en el sistema social de la organización. 

Discrepancias en tal sentido solo pueden conducir al fracaso. 

Una adecuada intervención de desarrollo organizacional puede reducir el impacto de tales cambios y posicionar a la organización en una plataforma de desarrollo superior.

Se pueden distinguir tres tipos de tecnologías a utilizar de acuerdo como interactúan entre los sistemas de recursos humanos y la infraestructura tecnológica:


\begin{itemize}
\item De unidad o de menor escala, en la que se elaboran productos o servicios hechos a la medida y gusto del consumidor. Por ejemplo, el trabajo de un arquitecto o la configuración de una red de computación.
\item De producción en gran escala como la que se encuentra en las líneas continuas de montaje. Ejemplos son las fábricas de automotores.
\item De proceso, que implica la transformación de materia prima a través de una serie continua de procesos automatizados. Por ejemplo, el trabajo que se realiza en una refinería de petróleo o en una central nuclear.
\end{itemize}


\section{Acerca del formato de este trabajo. Notas de Borrador}
Para este trabajo se utilizó la planilla "Quantum Hall effect report", recomendada para "Homework Assignment" de la Universidad de Copenhague.

El código fuente del documento se encuentra en:

https://www.overleaf.com/read/vknzvkjbdbbk

\begin{thebibliography}{9}

\bibitem{Naisbitt1984}
Naisbitt, J. \emph{Megatendencias Dieznuevas direcciones de cambio..}. 1984 Buenos Aires, Fundación CERIEN.

\bibitem{Toffler1997}
Toffler, \emph{A. La tercera ola}. 1997. Barcelona, Plaza y Janes. 


\bibitem{Packard1964}
Packard, V. \emph{The waste makers.} 1964. London. Penguin Books.

\end{thebibliography}
\end{document}



